% Thanks to Kevin D. McGrath for the presentation template.
\documentclass[xcolor={dvipsnames,svgnames},hyperref=dvips]{beamer}

% Packages.
\usepackage{graphicx}
\usepackage{amssymb}
\usepackage{amsmath}
\usepackage{amsthm}
\usepackage{url}
\usepackage{pstricks}
\usepackage{pst-node}

% Hyperlinks.
\def\name{Wade Cline}
\hypersetup{
  colorlinks = true,
  citecolor = black,
  linkcolor = black,
  urlcolor = black,
  pdfauthor = {\name},
  pdfkeywords = {``computer science'', lug, cryptography, full-disk},
  pdftitle = {Linux Full-Disk Encryption},
  pdfsubject = {Full-Disk Encryption Basics},
  pdfpagemode = UseNone
}

% ???
\usetheme[hideothersubsections]{Hannover}
\usecolortheme{sidebartab}

% Presentation metadata.
\title[Full-Disk Encryption]{}
\author{Wade Cline}
\date{04 March 2014}

\AtBeginSection[]
{
  \begin{frame}<beamer>{Outline}
    \tableofcontents[currentsection]
  \end{frame}
}
\AtBeginSubsection[]
{
  \begin{frame}<beamer>{Outline}
    % \transwipe
    \tableofcontents[currentsection,currentsubsection]
  \end{frame}
}

\begin{document}

% Title.
\begin{frame}
  \titlepage
\end{frame}

% Table of Contents.
\begin{frame}{Outline}
  % \transwipe
  \tableofcontents
  % You might wish to add the option [pausesections]
\end{frame}

\section{Basics}\label{section:basics}
	\subsection{Terminology}
	\begin{frame}
		\frametitle{Plaintext vs. Ciphertext}
		\begin{itemize}
		\item Plaintext: Unencrypted data.
		\item Ciphertext: Encrypted data.
		%TODO: Encryption and Decryption images.
		\end{itemize}
	\end{frame}

	\begin{frame}
		\frametitle{Symmetric Cryptography}
		\begin{itemize}
		\item Same key used to encrypt plaintext and to decrypt ciphertext. 
		\item Useful when only one person needs access to information.
		\item Relevant to today's discussion.
		\item Contrasted with Asymetric Cryptography.
		\item Examples: AES, Blowfish, 3DES, Serpent, Twofish.
		\end{itemize}
	\end{frame}

	\begin{frame}
		\frametitle{Asymetric Cryptography}
		\begin{itemize}
		\item AKA Public-Private Key Cryptography.
		\item Different keys used to encrypt plaintext and to decrypt ciphertext.
		\item Useful when communicating securely between two people.
		\item Contrasted with Symmetric Cryptography.
		\item Example: RSA.
		\end{itemize}
	\end{frame}

	\begin{frame}
		\frametitle{Block vs. Stream Ciphers}
		\begin{itemize}
		\item Block Ciphers operate on larged, fixed-length chunks of bits.
		\item Stream Ciphers encrypt digits one-at-a-time.
		\item Similar to Block vs. Character Devices.
		\item Block Ciphers relevant to today's discussion.
		\end{itemize}
	\end{frame}

	\subsection{Physical Media}
	\begin{frame}
		\frametitle{Disk Layout}
		\begin{itemize}
		\item Linux divides disks into 512-byte sections known as \textit{sectors}.
		\item Filesystem divides read/writes into \textit{blocksize} units (multiple of sector size).
		\item Encrypted on sector-level, though filesystem may do block-level writes.
		\item Block Cipher size $\leq$ Sector size $\leq$ Block size
			\begin{itemize}
			\item Common values: 128 \textit{bits} - 512 \textit{bytes} - 4096 bytes, respectively.
			\end{itemize}
		% TODO: Visualization.
		\end{itemize}
	\end{frame}

	\begin{frame}
		\frametitle{Wear-leveling}
		\begin{itemize}
		\item Attempts to distribute writes evenly across the device.
		\item Helps with device longevity.
		\item Used mostly in USB and SSD devices.
		\item Makes overwriting files difficult.
		\item Emphasizes proactive encryption.
		\end{itemize}
	\end{frame}

\section{Wiping}\label{wiping}
	\begin{frame}
		\frametitle{Naive Wiping}
		\begin{itemize}
		\item Nuke disk \textit{once} with a series of '0's (\texttt{/dev/zero}).
		\item Fast, simple, but leaks data.
		\item Each filesystem has a metaphorical ``fingerprint''.
		\item Labratory forensic analysis may reveal old data (discussed later).
		% TODO: Visualization/examples.
		\end{itemize}
	\end{frame}

	\begin{frame}
		\frametitle{Secure Wiping}
		\begin{itemize}
		\item Use cryptographic-grade pseudo-random numbers.
			\begin{itemize}
			\item \texttt{/dev/urandom} should provide a decent stream.
			\item \texttt{/dev/random} in theory better, but in practice you'd be dead before completion.
			\end{itemize}
		\item Wipe multiple times.
			\begin{itemize}
			\item In theory, proper analysis could reveal previous magnetic states.
			\item Government-``recommended'' is 7 times.
			\end{itemize}
		\item Depending on size of drive and computing power this could take days or weeks.
		\end{itemize}
	\end{frame}

\section{Encryption}\label{section:encryption}
	\subsection{Abstraction}
	\begin{frame}
		\frametitle{Ciphers}
		\begin{itemize}
		\item Transforms a chunk of plaintext into ciphertext.
		\item Examples: AES, Serpent, Twofish.
		\end{itemize}
	\end{frame}

	\begin{frame}
		\frametitle{Mode of Operation}
		\begin{itemize}
		\item Recall block ciphers work on fixed-length set of bits.
		\item Recall Linux sector size is 512 \textit{bytes}.
		\item Many of today's block ciphers are 128 \textit{bits}.
		\item A \textit{mode of operation} describes how to repeatedly use a block cipher to encrypt a larger area, such as a sector.
		\item Simply breaking apart reveals patterns in data.
		\item Examples: CBC, XTS.
		% TODO: Visualization.
		\end{itemize}
	\end{frame}

	\begin{frame}
		\frametitle{Initialization Vector}
		\begin{itemize}
		\item What is the first block chained with? An \textit{initialization vector}.
		\item Part of the Mode of Operation.
		\item Must appear ``random'' to avoid watermarking.
		\item Example: ESSIV.
		\end{itemize}
	\end{frame}

	\begin{frame}
		\frametitle{Keyfiles}
		\begin{itemize}
		\item Instead of passphrase, use a \textit{keyfile}.
		\item Not limited by human memory, fully-random numbers (\texttt{/dev/random}).
		\item For added protection, encrypt keyfile with passphrase.
			\begin{itemize}
			\item Can also protect against keylogging.
			\end{itemize}
		\end{itemize}
	\end{frame}

	\begin{frame}
		\frametitle{Key-stretching}
		\begin{itemize}
		\item Passphrases usually \textit{hashed} before input into algorithm.
		\item Increase number of hashing \textit{rounds} before input (0.5 - 2 seconds-worth).
		\item Slows down brute-force attacks dramatically.
		\end{itemize}
	\end{frame}
	
	%TODO: Salts.

	\subsection{Implementation}
	\begin{frame}
		\frametitle{\texttt{cryptsetup}}
		\begin{itemize}
		\item \texttt{cryptsetup} front-end to kernel crypto API.
		\item Two main modes: ``plain'' \texttt{dm-crypt} and LUKS (Linux Unified Key Setup).
		\item LUKS is feature-rich, \texttt{dm-crypt} is not.
		\item Detailed description of LUKS beyond scope of presentation.
		\item Simplfied: Beginners use LUKS, advanced users \textit{may} wish to use \texttt{dm-crypt}.
		\end{itemize}
	\end{frame}

	\begin{frame}
		\frametitle{Mappings}
		\begin{itemize}
		\item Rather than ``encrypt'' function, create abstraction layer over device.
		\item Entries go under the \texttt{/dev/mapper} directory.
		\item For example, use \texttt{/dev/mapper/root} to access an encrypted \texttt{/dev/sda}.
			\begin{itemize}
			\item \texttt{/dev/sda} appears as garbage (because it's encrypted) to anyone looking at it.
			\item \texttt{/dev/mapper/root} looks like a normal hard drive.
			\end{itemize}
		% TODO: Visualization.
		\end{itemize}
	\end{frame}

	\begin{frame}
		\frametitle{Multiple Layers}
		\begin{itemize}
		\item Why rely on one cipher that may get cracked?
		\item Make a mapping over the mapping.
			\begin{itemize}
			\item Example: \texttt{/dev/mapper/root -> /dev/mapper/.extra\_layer -> /dev/sda}
			\end{itemize}
		\item Keep adding layers for more security.
		\end{itemize}
	\end{frame}

\section{Early Userspace}\label{section:hell}
	\subsection{Early-Userspace hell}
	\begin{frame}
		\frametitle{Early-userspace Hell}
		\begin{itemize}
		\item No root fs, need to set it up.
		\item Create "early userspace" environment to mount root filesystem.
		\item Simply copying binaries unlikely to work (dynamically-linked).
			\begin{itemize}
			\item \texttt{ldd} shows dependencies.
			\item Copy all dependencies, or just make binaries statically-linked.
			\end{itemize}
		\item \texttt{busybox} highly-useful toolkit.
		\item Kernel will execute an \texttt{init} script.
		\item CPIO gzipped archive.
		\end{itemize}
	\end{frame}

% References?
\begin{frame}
\end{frame}

\end{document}

