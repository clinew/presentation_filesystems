% Thanks to Kevin D. McGrath for the presentation template.
\documentclass[xcolor={dvipsnames,svgnames},hyperref=dvips]{beamer}

% Packages.
\usepackage{graphicx}
\usepackage{amssymb}
\usepackage{amsmath}
\usepackage{amsthm}
\usepackage{url}
\usepackage{pstricks}
\usepackage{pst-node}

% Hyperlinks.
\def\name{Wade Cline}
\hypersetup{
  colorlinks = true,
  citecolor = black,
  linkcolor = black,
  urlcolor = black,
  pdfauthor = {\name},
  pdfkeywords = {``computer science'', lug, fs, filesystems},
  pdftitle = {Linux Filesystems for Windows Users,},
  pdfsubject = {Linux Filesystems Introduction},
  pdfpagemode = UseNone
}

% ???
\usetheme[hideothersubsections]{Hannover}
\usecolortheme{sidebartab}

% Presentation metadata.
\title[Linux Filesystems Introduction]{An Introduction to the Linux Filesystem for Windows Users}
\author{Wade Cline}
\date{12 November 2013}

\AtBeginSection[]
{
  \begin{frame}<beamer>{Outline}
    \tableofcontents[currentsection]
  \end{frame}
}
\AtBeginSubsection[]
{
  \begin{frame}<beamer>{Outline}
    % \transwipe
    \tableofcontents[currentsection,currentsubsection]
  \end{frame}
}

\begin{document}

% Title.
\begin{frame}
  \titlepage
\end{frame}

% Table of Contents.
\begin{frame}{Outline}
  % \transwipe
  \tableofcontents
  % You might wish to add the option [pausesections]
\end{frame}

\section{Moving Over From Windows}\label{section:windowsux}
	\subsection{Terminology}
	\begin{frame}
		\frametitle{Terminology}
		\begin{itemize}
		\item Binaries, not executables.
		\item Directories, not folders.
		\item Read, not load.
		\item Symbolic links, not shortcuts.
		\item Write, not save.
		\end{itemize}
	\end{frame}

	\subsection{Frequently Asked Questions}
	\begin{frame}
		\frametitle{``Where is \texttt{Program Files}?''}
		\begin{itemize}
		\item \texttt{/bin}, \texttt{/sbin}, \texttt{/usr/bin}, \texttt{/usr/sbin}, \texttt{/usr/local/bin}, \texttt{/usr/local/sbin}
		\item PATH environment variable
		\item \texttt{which} command
		\end{itemize}
	\end{frame}

	\begin{frame}
		\frametitle{``Where is my \texttt{Trash Bin}?''}
		\begin{itemize}
		\item Depends on the Desktop, not the filesystem.
		\item ``So then what's \texttt{Lost+Found}?''
			\begin{itemize}
			\item Lost blocks of the filesystem.
			\item Usually not an issue.
			\end{itemize}
		\end{itemize}
	\end{frame}

	\begin{frame}
		\frametitle{``Where is my data stored?''}
		\begin{itemize}
		\item \texttt{/home/<username>}
		\end{itemize}
	\end{frame}

	\begin{frame}
		\frametitle{``Where are my configuration files?''}
		\begin{itemize}
		\item Dot files.
		\item Within home directory \texttt{~/}.
		\end{itemize}
	\end{frame}

	\begin{frame}
		\frametitle{``Where does my USB stick appear?''}
		\begin{itemize}
		\item Raw file appears under \texttt{/dev}.
			\begin{itemize}
			\item \texttt{dmesg | tail} for the exact name.
			\end{itemize}
		\item Filesystem usually mounted under \texttt{/media}.
		\end{itemize}
	\end{frame}

	\begin{frame}
		\frametitle{``Where is my \texttt{C://} drive?"}
		\begin{itemize}
		\item Like USB sticks, under \texttt{/dev}.
		\item Usually mounted as root \texttt{/}.
		\end{itemize}
	\end{frame}

	\begin{frame}
		\frametitle{``How much space do I have/how much space am I using?''}
		\begin{itemize}
		\item Use \texttt{df} to see \underline{d}isk \underline{f}ree space.
		\item Use \texttt{du} to see \underline{d}isk \underline{u}sage.
		\item Default output is in bytes, \texttt{-h} for \underline{h}uman-readable output.
		\end{itemize}
	\end{frame}

	\begin{frame}
		\frametitle{``Any other questions?''}
		\begin{itemize}
		\item If you have a question then others may, too.
		\item Help me improve this guide!
		\end{itemize}
	\end{frame}

\section{Linux Filesystem Hierarchy}\label{section:lfsh}
	\subsection{The Three Tiers}
	\begin{frame}
		\frametitle{The Three Tiers}
		\begin{itemize}
		\item \texttt{/}, essential for system booting and mounting \texttt{/usr}.
		\item \texttt{/usr}, read-only system data for normal system operation.
		\item \texttt{/usr/local}, locally-installed software.
			\begin{itemize}
			\item Package managers usually install under \texttt{/} and \texttt{/usr}.
			\end{itemize}
		\end{itemize}
	\end{frame}

	\subsection{Common Directories}
	\begin{frame}
		\frametitle{Common Directories}
		\begin{itemize}
		\item \texttt{/bin}, \underline{bin}ary files.
		\item \texttt{/include}, header files for C/C++ Programs (\texttt{stdlib.h}, \texttt{stdio.h}, \texttt{string.h}, \&c.).
		\item \texttt{/lib}, \underline{lib}raries for programs.
		\item \texttt{/sbin}, binary files for root (\underline{s}uperuser \underline{bin}aries).
		\end{itemize}
	\end{frame}

	\subsection{Specific Directories}
	\begin{frame}
		\frametitle{Specific Directories}
		\begin{itemize}
		\item \texttt{/boot}, files essential for booting Linux.
			\begin{itemize}
			\item kernel, \texttt{initramfs}, \texttt{System.map}.
			\end{itemize}
		\item \texttt{/dev}, Virtual \texttt{devfs} filesystem that exports hardware devices as files.
			\begin{itemize}
			\item Hard Drives.
			\item CD drives.
			\item Floppy drives.
			\item \texttt{tty}s.
			\item More details later.
			\end{itemize}
		\item \texttt{/etc}, system-wide configuration files.
		\item \texttt{/home}, user's data.
		\end{itemize}
	\end{frame}

	\begin{frame}
		\frametitle{Specific Directories (cont.)}
		\begin{itemize}
		\item \texttt{/media}, USB sticks, flash cards, and other removable storage devices.
		\item \texttt{/mnt}, like \texttt{/media}, usually for System Administrators.
		\item \texttt{/opt}, ``Add-on application software packages''; usually not important.
		\item \texttt{/proc}, Virtual filesystem that exports system data.
			\begin{itemize}
			\item \texttt{/proc/cpuinfo}, processor information.
			\item \texttt{/proc/meminfo}, memory information.
			\item \texttt{/proc/version}, Linux version and distribution information.
			\item Some files may be ineracted with.
			\item \texttt{echo 3 > /proc/sys/vm/drop\_caches} to drop caches.
			\end{itemize}
		\end{itemize}
	\end{frame}
		
	\begin{frame}
		\frametitle{Specific Directories (cont.)}
		\begin{itemize}
		\item \texttt{/root}, home directory for root.
		\item \texttt{/run}, volatile information accumulated since system boot; usually not important.
		\item \texttt{/sys}, Virtual filesystem, exports kernel objects.
		\item \texttt{/tmp}, Temporary files for programs; usually not important.
		\item \texttt{/var}, Data that varies over time (system logs, mail, \&c.).
		\item \texttt{/usr/share}, Architecture-independent, read-only data (\texttt{lspci}).
		\item \texttt{/usr/src}, kernel source code.
		\end{itemize}
	\end{frame}

\section{Power User Topics}\label{section:poweruser}
	\begin{frame}
		\frametitle{File types}
		\begin{itemize}
		\item Regular files.
		\item Block files.
		\item Character files.
		\item Socket files.
		\item \texttt{ls -l}
		\end{itemize}
	\end{frame}

	\begin{frame}
		\frametitle{Links}
		\begin{itemize}
		\item Symbolic.
		\item Hard.
		\item \texttt{stat}
		\item \texttt{ls -l}
		\end{itemize}
	\end{frame}

	\begin{frame}
		\frametitle{\texttt{devfs}}
		\begin{itemize}
		\item \texttt{sd*}
		\item \texttt{sr*}
		\item \texttt{/dev/null}
		\item \texttt{/dev/random}
		\item \texttt{/dev/urandom}
		\item \texttt{/dev/zero}
		\end{itemize}
	\end{frame}

	\begin{frame}
		\frametitle{Formatting Filesystems}
		\begin{itemize}
		\item \texttt{mkfs}
		\end{itemize}
	\end{frame}

	\begin{frame}
		\frametitle{Mounting Filesystems}
		\begin{itemize}
		\item \texttt{mount}
		\item \texttt{-t type}
		\item \texttt{-o options}
		\item \texttt{device path}
		\item \texttt{mount point}
		\end{itemize}
	\end{frame}

	\begin{frame}
		\frametitle{Loopback Devices}
		\begin{itemize}
		\item File as a block device.
		\item \texttt{/dev/loop*}
		\item \texttt{losetup}
		\end{itemize}
	\end{frame}

	\begin{frame}
		\frametitle{Metadata vs. Data}
		\begin{itemize}
		\item Metadata used within the filesystem for its own purposes.
		\item Data refers to files within the filesystem.
		\end{itemize}
	\end{frame}

	\begin{frame}
		\frametitle{Filesystem Consistency}
		\begin{itemize}
		\item Consistent if all metadata is intact.
		\item \texttt{fsck}, \underline{f}ile\underline{s}ystem \underline{c}onsistency chec\underline{k}.
		\end{itemize}
	\end{frame}

	\begin{frame}
		\frametitle{Journaling}
		\begin{itemize}
		\item Filesystem \emph{consistency} tool; protections against system freezes, power outages, \&c.
		\item \emph{Replaying} the journal.
		\item \texttt{ext3}'s three modes of journaling:
			\begin{itemize}
			\item \texttt{journal}: Data and metadata to journal.
			\item \texttt{ordered}: Data updates to filesystem, then metadata comitted to journal.
			\item \texttt{writeback}: Metadata comitted to journal, possibly before data updates.
			\end{itemize}
		\end{itemize}
	\end{frame}

	\begin{frame}
		\frametitle{Block size}
		\begin{itemize}
			\item Size of chunks allocated for files.
		\end{itemize}
	\end{frame}

	\begin{frame}
		\frametitle{\texttt{dd}}
		\begin{itemize}
			\item \underline{D}isk \underline{d}uplicator (or \underline{d}isk \underline{d}ump).
			\item \texttt{if=<path>}, \underline{i}nput \underline{f}ile.
			\item \texttt{of=<path>}, \underline{o}output \underline{f}ile.
			\item \texttt{bs=<size>}, block size.
			\item \texttt{count=<size>}, number of block to transfer.
		\end{itemize}
	\end{frame}

	\begin{frame}
		\frametitle{\texttt{/etc/fstab}}
		\begin{itemize}
			\item Which filesystems get mounted where.
			\item device\_file mount\_point filesystem\_type options dump check
		\end{itemize}
	\end{frame}

	\begin{frame}
		\frametitle{\texttt{/etc/mtab}}
		\begin{itemize}
			\item Which filesystems are currently mounted where.
			\item Format usually same as \texttt{/etc/fstab}.
		\end{itemize}
	\end{frame}

% References
\section{References}

\begin{frame}
	\frametitle{References}
	\begin{itemize}
	\item ``Filesystem Hierarchy Standard''. Rusty Russell, Christopher Yeoh, and Dan Quinlan. Version 2.3. \texttt{http://refspecs.linuxfoundation.org/fhs.shtml}
	\end{itemize}
\end{frame}

\end{document}

